\documentclass{beamer}
\usetheme{metropolis}

\usepackage{fontspec}
\usepackage{polyglossia}
\setmainlanguage{english}

\usepackage[useregional]{datetime2}
\usepackage{listings}
\usepackage{lstautogobble}
\lstset{autogobble,basicstyle=\ttfamily,showstringspaces=alse}

\usepackage{enumitem}
\setitemize{label=\usebeamerfont*{itemize item}%
  \usebeamercolor[fg]{itemize item}
  \usebeamertemplate{itemize item}}

\title{Haskell Development with Nix, GHCJS and Miso}
\date{\DTMdisplaydate{2018}{07}{09}{}}
\author{Moritz Kiefer (\texttt{@cocreature})}

\metroset{block=fill}

\begin{document}
\maketitle

\begin{frame}{Terminology}
  \begin{block}{Nix}
    Language and package manager
  \end{block}
  \begin{block}{NixOS}
    Linux distribution built on top of Nix
  \end{block}
  \begin{block}{nixpkgs}
    Official repository for nix expressions
  \end{block}
\end{frame}

\begin{frame}{The Life of a Package}
  \begin{itemize}
  \item Everything happens in the Nix \emph{store} (\texttt{/nix/store})
  \item Nix \emph{expressions} produce Nix \emph{derivations}
  \item A Nix derivation consists of
    \begin{itemize}
    \item Input derivations
    \item Output paths
    \item Build instructions
    \end{itemize}
  \item \emph{Realising} a derivation ensures that the output paths are valid
    \begin{itemize}
    \item Either by fetching them from a \emph{substitute}
    \item or by building the derivation
    \end{itemize}
  \end{itemize}
\end{frame}

\section{Nix Language}
\begin{frame}{Basic Values}
  \begin{itemize}
  \item Booleans
  \item Integers
  \item Floating points
  \item Null
  \item Paths
  \end{itemize}
\end{frame}

\begin{frame}[fragile]{Strings}
  \begin{block}{Normal String Literals}
    \begin{lstlisting}
      "foo bar"
    \end{lstlisting}
  \end{block}
  \begin{block}{Antiquotation}
    \begin{lstlisting}
      "foo ${var} bar"
    \end{lstlisting}
  \end{block}
  \begin{block}{Indented Strings}
    \begin{lstlisting}
    ''
         foo
         bar
    ''
    \end{lstlisting}
  \end{block}
\end{frame}

\begin{frame}[fragile]{Lists}
  \begin{itemize}
  \item Enclosed in square brackets
  \item Items separated by spaces
  \end{itemize}
  \begin{block}{Example}
    \begin{lstlisting}
      [ 123 ./foo.nix "abc" ]
    \end{lstlisting}
  \end{block}
\end{frame}

\begin{frame}[fragile]{Sets}
  \begin{block}{Example}
    \begin{lstlisting}
      { a = 1; b = 5; }
    \end{lstlisting}
  \end{block}
  \begin{block}{Attribute Selection}
    \begin{lstlisting}
      { a = 1; b = 5; }.a
    \end{lstlisting}
  \end{block}
  \begin{block}{Recursive Sets}
    \begin{lstlisting}
      rec {
        x = y;
        y = 123;
      }
    \end{lstlisting}
  \end{block}
\end{frame}

\begin{frame}[fragile]{\texttt{let}-expressions}
  \begin{block}{Example}
    \begin{lstlisting}
      let
        x = "foo";
        y = "bar";
      in x + y
    \end{lstlisting}
  \end{block}
\end{frame}

\begin{frame}[fragile]{Inheriting Attributes}
  \begin{block}{Without \texttt{inherit}}
    \begin{lstlisting}
      let x = 123; in
      { x = x;
      }
    \end{lstlisting}
  \end{block}
  \begin{block}{With \texttt{inherit}}
    \begin{lstlisting}
      let x = 123; in
      { inherit x;
      }
    \end{lstlisting}
  \end{block}
\end{frame}

\begin{frame}[fragile]{Functions}
  \begin{block}{General Form}
    \emph{pattern}: body
  \end{block}
  \begin{block}{Set Pattern}
    \begin{lstlisting}
      {x, y}: x + y
    \end{lstlisting}
  \end{block}
  \begin{block}{Default Values}
    \begin{lstlisting}
      {x, y ? 0}: x + y
    \end{lstlisting}
  \end{block}
  \begin{block}{Additional Arguments}
    \begin{lstlisting}
      {x, y, ...}: x + y
    \end{lstlisting}
  \end{block}
\end{frame}

\begin{frame}[fragile]{\texttt{with}-expressions}
  Brings all attributes of a set in scope

  \begin{block}{Example}
    with {x = 1; y = 2}; x
  \end{block}
\end{frame}

\section{CLI Interface}
\begin{frame}{CLI Overview}
  \begin{description}[leftmargin=!,labelwidth=\widthof{\bfseries nix-instantiate}]
  \item[nix-instantiate]
    Create a derivation from a nix expression

  \item[nix-store]
    Manipulate and query the nix store

  \item[nix-build]
    Create a derivation from a nix expression and build it

  \item[nix-shell]
    Setup shell environment for building a derivation
  \end{description}
\end{frame}

\begin{frame}{nix-instantiate}
  \begin{itemize}
  \item Evaluates nix expression and prints path of derivation
  \item Defaults to \texttt{default.nix}
  \item Select attribute with \texttt{-A}
  \item Pass arbitrary expression with \texttt{-E}
  \end{itemize}
\end{frame}

\begin{frame}{nix-store}
  \begin{itemize}
  \item Build a derivation using \texttt{--realise}
  \item Garbage collect the nix store using \texttt{--gc}
  \item Show immediate dependencies using \texttt{--query --references}
  \item Show transitive dependencies using \texttt{--query --requisites}
  \item Show referrers of store path using \texttt{--query --referrers}
  \end{itemize}
\end{frame}

\begin{frame}{nix-build}
  \begin{itemize}
  \item Combines \texttt{nix-instantiate} and \texttt{nix-store --realise}
  \item Symlinks store path to \texttt{result}
  \end{itemize}
\end{frame}

\begin{frame}{nix-shell}
  \begin{itemize}
  \item Opens shell that has dependencies of derivation in scope
  \item Clear environment using \texttt{--pure}
  \item Specify packages that should be in scope using \texttt{-p}
  \end{itemize}
\end{frame}

\section{Miso}
\begin{frame}[fragile]{\texttt{App} type}
  \begin{lstlisting}[language=Haskell]
    data App model action = App
      { model :: model
      , update :: action -> model
               -> Effect action model
      , view :: model -> View action
      , subs :: [ Sub action ]
      , events :: M.Map MisoString Bool
      , initialAction :: action
      , mountPoint :: Maybe MisoString
      }
  \end{lstlisting}
\end{frame}

\begin{frame}[fragile]{Effect}
  \begin{lstlisting}[language=Haskell]
    data Effect action model =
      Effect model [(action -> IO ()) -> IO ()]

    noEff :: model -> Effect action model
    noEff m = Effect m []

    (<#) :: model -> IO action -> Effect action model
    m <# act = Effect m [\sink -> sink =<< a]
  \end{lstlisting}
\end{frame}

\begin{frame}[fragile]{View}
  \begin{lstlisting}[language=Haskell]
    div_
      [ class_ "foobar" ]
      [ button_
          [ onClick ButtonClicked ]
          [ "Click me!" ]
      ]
  \end{lstlisting}
\end{frame}

\begin{frame}[fragile]{MisoString}
  \begin{itemize}
  \item Type synonym for
    \begin{itemize}
    \item \texttt{JSString} when compiled with \texttt{GHCJS}
    \item \texttt{Text} when compiled with \texttt{GHC}
    \end{itemize}
  \item Conversion using \texttt{ToMisoString} typeclass
    \begin{lstlisting}[language=Haskell]
      class ToMisoString str where
        toMisoString :: str -> MisoString
        fromMisoString :: MisoString -> str
    \end{lstlisting}
  \end{itemize}
\end{frame}
\end{document}